\documentclass[a4paper,10pt]{article} \usepackage{anysize}
\marginsize{2cm}{2cm}{1cm}{1cm}
%\textwidth 6.0in \textheight = 664pt
\usepackage{xltxtra}
\usepackage{xunicode} \usepackage{graphicx}
\usepackage{color} \usepackage{xgreek} \usepackage{fancyvrb}
\usepackage{minted}
\usepackage{listings}
\usepackage{enumitem} \usepackage{framed} \usepackage{relsize}
\usepackage{float} \setmainfont[Mapping=TeX-text]{CMU Serif}
\begin{document}

\begin{titlepage}
\begin{center}
\begin{figure}[t] 
     \includegraphics[scale=0.7]{title/ntua_logo}
\end{figure}
\begin{LARGE}\textbf{ΕΘΝΙΚΟ ΜΕΤΣΟΒΙΟ ΠΟΛΥΤΕΧΝΕΙΟ\\}\end{LARGE}
\vspace{2cm}
\begin{Large}
ΣΧΟΛΗ ΗΜ\&ΜΥ\\
Λειτουργικά Συστήματα
1\textsuperscript{η} Άσκηση\\
Ακ. έτος 2011-2012\\
\end{Large}
\vspace{5cm}
\Large Τμήμα Β, Ομάδα 3\textsuperscript{η}\\
\vspace{1cm}
\begin{tabular}{l r}
\Large{Γερακάρης Βασίλης}&
\large{Α.Μ.: 03108092}\\
\Large{Λύρας Γρηγόρης}&
\large{Α.Μ.: 03109687}\\
\end{tabular}\\
\vspace{5cm}

\vfill
\large\today\\
\end{center}
\end{titlepage}



\section*{} \setcounter{section}{1}
\subsection{Υλοποίηση σημαφόρων με σωληνώσεις του UNIX} 
Ο πηγαίος κώδικας της pipesem.c που γράψαμε είναι ο παρακάτω:
\inputminted[linenos,fontsize=\footnotesize,frame=leftline]{c}{files/pipesem.c}

\newpage
\emph{Ερωτήσεις}
\begin{enumerate}
	\item Στο πρόγραμμα τρέχουν 2 διεργασίες. Η αρχική διεργασία (πατέρας)
		κάνει fork το παιδί και έπειτα κάνει wait στο σημαφόρο.  Το παιδί,
		αφού δημιουργηθεί, κάνει sleep για 5 δευτερόλεπτα. Επειτα κάνει signal
		στο σημαφόρο, ώστε να ξεμπλοκάρει τον πατέρα που περιμένει, και
		τερματίζει.
	\item H pipesem.c που γράψαμε, έχει ως βασική αρχή λειτουργίας ότι όταν
		μια διεργασία (ή νήμα) προσπαθεί να διαβάσει από το read άκρο ενός
		άδειου pipe μπλοκάρει εκεί μέχρι να βρεθούν δεδομένα προς ανάγνωση.
		Αντίστοιχα, όταν μια διεργασία εξέλθει από το critical section,
		τροφοδοτεί το write άκρο του pipe με 1 token (int) ώστε να μπορέσει η
		επόμενη διεργασία να εισέλθει.
	\item Θα μπορούσε, μιας και εξαρτάται από το πως το ΛΣ διαχειρίζεται της
		σωληνώσεις. Στην περίπτωσή μας, γνωρίζοντας ότι το UNIX αντιμετωπίζει
		τα read requests στο άκρο του pipe ως μία FIFO λίστα, είμαστε σίγουροι
		ότι δε θα οδηγηθεί καμία διεργασία σε λιμοκτονία.\\
\end{enumerate}


\subsection{Παράλληλος υπολογισμός του συνόλου Mandelbrot} 
Ο πηγαίος κώδικας παρατίθεται παρακάτω:
\inputminted[linenos,fontsize=\footnotesize,frame=leftline]{c}{files/mandel.c}

\begin{figure}[H]
	\centering
	\includegraphics[width=0.7\textwidth]{files/mandel.png}
	\caption{Screenshot}
\end{figure}

\emph{Ερωτήσεις}
\begin{enumerate}
	\item Χρησιμοποιείται ένας σημαφόρος για κάθε διεργασία, ουσιαστικά ένα
		κυκλικό σχήμα με διεργασίες και τα pipes που τις ενώνουν.
	\item Η έξοδος που προέκυψε κατά την εκτέλεση ήταν η εξής, με τους
		αντίστοιχους χρόνους με και χωρίς παραλληλία (οι χρόνοι προέκυψαν σε
		μηχάνημα με 4 πυρήνες:\\
		Mε μία διεργασία:\\
		\begin{minted}{sh}
		./mandel  0.69s user 0.01s system 99% cpu 0.705 total
		\end{minted}
		Με 30 διεργασίες:\\
		\begin{minted}{sh}
		./mandel  0.82s user 0.02s system 332% cpu 0.252 total
		\end{minted}
	\item Παρατηρούμε μία βελτίωση της τάξης του 65\% όταν ο υπολογισμός
		γίνεται παράλληλα. Το κρίσιμο τμήμα αποτελείται μόνο από τη φάση
		εξόδου, μιας και ο υπολογισμός είναι atomic και δεν επηρρεάζει κάπως
		τις άλλες διεργασίες ή το τελικό αποτέλεσμα.
	\item Αν δωθεί interrupt signal (\^\ C), το χρώμα του τερματικού θα
		παραμείνει στο χρώμα του χαρακτήρα που τυπώθηκε τελευταίος, με
		συνέπεια να χρειάζεται η εντολή "reset" για να επαναφερθεί στα
		κανονικά χρώματα. Για να αντιμετωπιστεί αυτό, θα πρέπει στο αρχείο
		mandel.c να τοποθετήσουμε ένα signal handler, που στην περίπτωση kill
		ή interrupt signal θα εκτελεί την "reset\_xterm\_color(1)" πρίν
		τερματίσει.
\end{enumerate}

\pagebreak

\subsection{Ταυτόχρονη πρόσβαση σε μοιραζόμενους πόρους} 
Το πρόγραμμα υλοποιήθηκε μόνο με χρήση pipesem\_wait() και pipesem\_signal(),
όπως ζητήθηκε.
\noindent Ο πηγαίος κώδικας παρατίθεται παρακάτω:

\inputminted[linenos,fontsize=\footnotesize,frame=leftline]{c}{files/procs-shm.c}
\newpage

\emph{Ερωτήσεις}
\begin{enumerate}
	\item Η create\_shared\_memory\_area(sizeof(int)), όπως δηλώνει και το
		όνομά της, δημιουργεί ένα χώρο στη μνήμη στον οποίο θα έχουν πρόσβαση
		και οι 3 διεργασίες.
	\item Το πρόγραμμά μας καλύπτει τη γενική περίπτωση (Β: n = n - K,  K
		defined) με ένα επαναληπτικό σχήμα στα pipesem\_signal(semA) και
		pipesem\_wait(semB).
\end{enumerate}

\subsection{Προαιρετικές ερωτήσεις (bonus)} 

\begin{enumerate}

	\item To πρόγραμμα αυτό δημιουργεί τόσες διεργασίες όσες του δωθούν σαν
		όρισμα (ή 10 αν δε δωθεί όρισμα) όπου η κάθε μία τυπώνει το αποτέλεσμα
		της \emph{rand()}. Πιθανώς να περίμενε κάποιος τα αποτελέσματα να είναι
		διαφορετικά (λόγω της τυχαιότητας της \emph{rand()}), όμως αυτό δεν ισχύει,
		γιατί όλες οι διεργασίες χρησιμοποιούν το ίδιο seed για τη γεννήτρια
		τυχαίων αριθμών και όλες επιστρέφουν το 1ο αποτέλεσμα.  Ένας τρόπος να
		έχουμε διαφορετικά τυχαία αποτελέσματα είναι πριν την εκτύπωση (δηλαδή
		αφού έχει δημιουργηθεί το κάθε παιδί), να δώσουμε την εντολή
		\emph{srand(time(NULL)*i)}, ώστε κάθε παιδί να χρησιμοποιήσει
		διαφορετικό seed.   
\end{enumerate}

\subsection{Bonus track: \\Οπτικοποίηση Mandelbrot Set με export σε εικόνα \&
\\επέκταση παραλληλοποίησης σε πολλαπλούς υπολογιστές (νησιά)}
\noindent Κυρίως κώδικας:

\inputminted[linenos,fontsize=\footnotesize,frame=leftline]{c}{files/mandel_parallel.c}

\noindent Bash script για την παραλληλοποίηση:

\inputminted[linenos,fontsize=\footnotesize,frame=leftline]{bash}{files/runem.sh}
\begin{figure}[H]
	\centering
	\includegraphics[width=0.7\textwidth]{files/mandel_parallel.png}
	\caption{Image Output}
\end{figure}


\end{document}

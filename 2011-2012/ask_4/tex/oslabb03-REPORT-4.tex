\documentclass[a4paper,10pt]{article} \usepackage{anysize}
\marginsize{2cm}{2cm}{1cm}{1cm}
%\textwidth 6.0in \textheight = 664pt
\usepackage{xltxtra}
\usepackage{xunicode}
\usepackage{graphicx}
\usepackage{color}
\usepackage[usenames,dvipsnames]{xcolor}
\usepackage{xgreek}
\usepackage{fancyvrb}
\usepackage{minted}
\usepackage{listings}
\usepackage{enumitem} \usepackage{framed} \usepackage{relsize}
\usepackage{float} \setmainfont[Mapping=TeX-text]{FreeSerif}
\begin{document}

\renewcommand{\theenumi}{\roman{enumi}}
\begin{titlepage}
\begin{center}
\begin{figure}[t] 
     \includegraphics[scale=0.7]{title/ntua_logo}
\end{figure}
\begin{LARGE}\textbf{ΕΘΝΙΚΟ ΜΕΤΣΟΒΙΟ ΠΟΛΥΤΕΧΝΕΙΟ\\}\end{LARGE}
\vspace{2cm}
\begin{Large}
ΣΧΟΛΗ ΗΜ\&ΜΥ\\
Λειτουργικά Συστήματα
1\textsuperscript{η} Άσκηση\\
Ακ. έτος 2011-2012\\
\end{Large}
\vspace{5cm}
\Large Τμήμα Β, Ομάδα 3\textsuperscript{η}\\
\vspace{1cm}
\begin{tabular}{l r}
\Large{Γερακάρης Βασίλης}&
\large{Α.Μ.: 03108092}\\
\Large{Λύρας Γρηγόρης}&
\large{Α.Μ.: 03109687}\\
\end{tabular}\\
\vspace{5cm}

\vfill
\large\today\\
\end{center}
\end{titlepage}


\section*{Ερωτήσεις}
\setcounter{section}{3}
\subsection{Άσκηση 1.1}
\subsubsection{}
Στην περίπτωση που έρθει ένα σήμα κατά την εκτέλεση της ρουτίνας ενός signal
handler, αυτό αγνοείται προσωρινά, αφού γίνονται masked όταν ένας από τους 2
τρέχει. Στο χρονοδρομολογητή χώρου πυρήνα οι αλλαγές γίνονται μέσα στον χώρο
του πυρήνα, οπότε τα σήματα που καταφτάνουν αυτόματα αναστέλλονται μέχρι να
επιστραφεί ο χειρισμός στο χώρο χρήστη.
\subsubsection{}
Τα SIGCHLD συνήθως προέρχονται από running παιδιά που δέχονται SIGSTOP από τον
scheduler όταν τελειώσει το κβάντο χρόνου τους (οπότε φέρνουμε για εκτέλεση
την επόμενη διεργασία) ή SIGCONT (όταν μια ready διεργασία επιλεγεί για
εκτέλεση, που το αγνοούμε). Επίσης έρχεται 1 σήμα SIGCHLD (κατα τη δημιουργία
μιας διεργασίας, όπου κάνει raise SIGSTOP στον εαυτό της) και 1 όταν κάνει
έξοδο.

Αν αλλάξει το status μιας διεργασίας-παιδί θα σταλεί SIGCHLD στον scheduler.
Στον signal handler του SIGCHLD, ελέγχουμε το αντίστοιχο macro για το
status (WIFEXITED, WIFSTOPPED, WIFCONTINUED, WIFSIGNALED) που πήραμε από τη
wait(pid), και δρούμε ανάλογα (harvest \& αφαίρεση από την ουρά αν χρειαστεί).
\subsubsection{}
Μια διεργασία έχει τη δυνατότητα να πιάσει το σήμα SIGSTΟP. Αν συμβεί αυτό,
ένας scheduler που χρησιμοποιεί μόνο SIGALRM και δεν ελέγχει αν η διεργασία
που προσπαθεί να σταματήσει όντως σταμάτησε, είναι δυνατόν να φτάσει σε
κατάσταση όπου περισσότερες από μία διεργασίες προσπαθούν να τρέξουν. Επίσης
το SIGCHLD χρησιμοποιείται για να ειδοποιηθεί ο scheduler σε περίπτωση που
κάποιο παιδί του πεθάνει (από κάποιο σήμα είτε από κανονική έξοδο).

\subsection{Άσκηση 1.2}
\subsubsection{}
Εμφανίζεται σχεδόν πάντα ο φλοιός ως current\_proc, αφού προκειμένου να
εκτελέσει τις εντολές που του δώθηκαν (μαζί και την 'p'), πρέπει να είναι
running. Ο λόγος που συμβαίνει αυτό είναι ότι οι εντολές του εκτελούνται
μέσα στο κβαντο χρόνου που του αντιστοιχεί. Υπάρχει μια απειροελάχιστη
πιθανότητα (αφού η επικοινωνία shell - scheduler γίνεται με χρήση pipes), το
shell να γράψει στο pipe και να εκπνεύσει αμέσως μετά το κβάντο χρόνου του,
επομένως θα εμφανιστεί η επόμενη διεργασία.
\subsubsection{}
Αν δε γινόταν απενεργοποιήση των σημάτων πριν την υλοποίηση των συναρτήσεων
του φλοιού θα ήταν πιθανό να έρθει ένα σήμα που θα μεταβάλει την κατάσταση της
ουράς εκτέλεσης την ώρα που ο φλοιός (για παράδειγμα) διέγραφε μια διεργασία.
Αυτό το race condition μπορούσε να προκαλέσει σφάλματα όπως διαγραφή λάθος
διεργασίας, ή (χειρότερα) κατάρρευση της ουράς με λάθος ανάθεση των pointers
στις επόμενες διεργασίες.

\subsection{Άσκηση 1.3}
\subsubsection{}
Θα ήταν δυνατόν να λιμοκτονήσει μια low\_priority διεργασία αν συνεχώς
δημιουργούσαμε και θέταμε ως high priority διεργασίες. Αφού η λίστα των υψηλής
προτεραιότητας διεργασιών θα είχε πάντα στοιχεία, ο χρονοδρομολογητής δε θα
επέλεγε ποτε τις διεργασίες με χαμηλή προτεραιότητα.

\section*{Προαιρετικές Ερωτήσεις}
\setcounter{section}{4}
\setcounter{subsection}{0}
\subsection{}
Η επικοινωνία γίνεται με τη χρήση σωληνώσεων (pipes). Ο πατέρας (scheduler)
περιμένει στο άκρο ανάγνωσης του pipe του παιδιού για να λάβει κάποια αίτηση
(request) όπως αυτή ορίζεται στο request.h. Όταν τη λάβει, απενεργοποιεί τα σήματα,
και την εξυπηρετεί. Στη συνέχεια ενεργοποιεί και πάλι τα σήματα και γράφει στο άκρο
εγγραφής του δικού του pipe. Στην άλλη μεριά το shell περιμένει στο άκρο
ανάγνωσης του pipe του πατέρα για να λάβει την απόκριση του scheduler.
Αφού τη λάβει συνεχίζει κανονικά τη λειτουργία του.

\subsection{}
Θα ήταν δυνατόν να διατηρείται στη λίστα με τις low\_priority διεργασίες ένας
counter που θα δείχνει πόσα κβάντα χρόνου έχουν περάσει από την δημιουργία
τους, ενώ υπάρχουν high\_priority διεργασίες. Ο counter αυτός θα αυξανόταν μέσα
στον SIGALRM handler (δίνοντας βέβαια αρκετή καθυστέρηση στο πρόγραμμα). Όταν
μία διεργασία έφτανε ένα προκαθορισμένο αριθμό στον counter, θα ανέβαινε σε
high\_priority ώστε να αρχίζει να εκτελείται.
\subsection{}
Το πρόβλήμα είναι ότι η L μένει μέσα στο critical section, αφού έχει μηδενίσει
το σημαφόρο. Όταν δημιουργηθούν οι Μ και H, η L δεν τρέχει λόγω
προτεραιότητας. Η H θα μπλοκάρει στον άδειο σημαφόρο και θα βγεί από την ουρά
εκτέλεσης των διεργασιών και θα τρέξει η M που κάνει infinite loop. Έτσι, το
σύστημα έχει μπλοκάρει σε μια κατάσταση όπου η L δεν μπορεί να τρέξει για να
απελευθερώσει το σημαφόρο, με αποτέλεσμα να κάνει μόνιμο wait η H σε αυτόν και
τρέχει μόνο η M.

Υπάρχουν διάφοροι τρόποι αντιμετώπισης αυτού του συμβάντος
\begin{itemize}
\item Θα μπορούσαμε να χρησιμοποιήσουμε ένα μηχανισμό γήρανσης, όπως
αναφέρεται στο παραπάνω υποερώτημα. Με τον τρόπο αυτό η L θα έπαιρνε κάποια
στιγμή χρόνο εκτέλεσης και θα απελευθέρωνε το σημαφόρο, ξεμπλοκάρωντας το
σύστημα.
\item Τέλος, μπορεί να αποθηκεύονται κάπου οι διεργασίες που βρίσκονται μέσα
σε critical sections και να τους δίνεται χρόνος εκτέλεσης κάθε x time
quantums, ανεξάρτητα με την προτεραιότητά τους, δίνοντάς τους έτσι χρόνο να
εξέλθουν από εκεί.

\end{itemize}

\vspace{1cm}
\def\thesubsection {Άσκηση \arabic{section}.\arabic{subsection}}
\section*{Πηγαίος κώδικας ασκήσεων}
\emph{Κοινή βιβλιοθήκη για διαχείρηση διπλά συνδεδεμένης, κυκλικής λίστας}
\inputminted[linenos,fontsize=\footnotesize,frame=leftline]{c}{files/queue.h}
\inputminted[linenos,fontsize=\footnotesize,frame=leftline]{c}{files/queue.c}
\setcounter{section}{1}
\setcounter{subsection}{0}
\subsection{}
\inputminted[linenos,fontsize=\footnotesize,frame=leftline]{c}{files/scheduler_1.1.c}
\subsection{}
\inputminted[linenos,fontsize=\footnotesize,frame=leftline]{c}{files/scheduler-shell_1.2.c}
\subsection{}
\inputminted[linenos,fontsize=\footnotesize,frame=leftline]{c}{files/scheduler-shell_1.3.c}

\section*{Παραδείγματα εκτέλεσης των προγραμμάτων}
\def\thesubsection {Πρόγραμμα \arabic{section}.\arabic{subsection}:}
\setcounter{subsection}{0}
\subsection{Scheduler}
\lstset{emph={PID, NMSG, delay, id, Shell},emphstyle=\color{blue}}
\lstset{emph={[2]SIREN, DEAD, DIE},emphstyle={[2]\color{BurntOrange}}}
\lstset{emph={[3]STOP},emphstyle={[3]\color{red}}}
\lstset{emph={[4]NEXT, NEEEEEEEEEEXT},emphstyle={[4]\color{ForestGreen}}}
\lstinputlisting[numbers=left,numberstyle=\tiny,basicstyle=\footnotesize\ttfamily,breaklines=true]{files/scheduler_1.1.out}
%\inputminted[linenos,fontsize=\footnotesize,frame=leftline]{bash}{files/scheduler_1.1.out}
\subsection{Scheduler with shell}
\lstinputlisting[numbers=left,numberstyle=\tiny,basicstyle=\footnotesize\ttfamily,breaklines=true]{files/scheduler-shell_1.2.out}
%\inputminted[linenos,fontsize=\footnotesize,frame=leftline]{bash}{files/scheduler-shell_1.2.out}
\subsection{Scheduler with shell \& priorities}
\lstinputlisting[numbers=left,numberstyle=\tiny,basicstyle=\footnotesize\ttfamily,breaklines=true]{files/scheduler-shell_1.3.out}
%\inputminted[linenos,fontsize=\footnotesize,frame=leftline]{bash}{files/scheduler-shell_1.3.out}

\end{document}

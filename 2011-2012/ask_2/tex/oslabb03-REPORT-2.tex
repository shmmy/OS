\documentclass[a4paper,10pt]{article} \usepackage{anysize}
\marginsize{2cm}{2cm}{1cm}{1cm}
%\textwidth 6.0in \textheight = 664pt
\usepackage{xltxtra}
\usepackage{xunicode} \usepackage{graphicx}
\usepackage{color} \usepackage{xgreek} \usepackage{fancyvrb}
\usepackage{minted}
\usepackage{listings}
\usepackage{enumitem} \usepackage{framed} \usepackage{relsize}
\usepackage{float} \setmainfont[Mapping=TeX-text]{CMU Serif}
\begin{document}

\begin{titlepage}
\begin{center}
\begin{figure}[t] 
     \includegraphics[scale=0.7]{title/ntua_logo}
\end{figure}
\begin{LARGE}\textbf{ΕΘΝΙΚΟ ΜΕΤΣΟΒΙΟ ΠΟΛΥΤΕΧΝΕΙΟ\\}\end{LARGE}
\vspace{2cm}
\begin{Large}
ΣΧΟΛΗ ΗΜ\&ΜΥ\\
Λειτουργικά Συστήματα
1\textsuperscript{η} Άσκηση\\
Ακ. έτος 2011-2012\\
\end{Large}
\vspace{5cm}
\Large Τμήμα Β, Ομάδα 3\textsuperscript{η}\\
\vspace{1cm}
\begin{tabular}{l r}
\Large{Γερακάρης Βασίλης}&
\large{Α.Μ.: 03108092}\\
\Large{Λύρας Γρηγόρης}&
\large{Α.Μ.: 03109687}\\
\end{tabular}\\
\vspace{5cm}

\vfill
\large\today\\
\end{center}
\end{titlepage}



\section*{} \setcounter{section}{1}
\subsection{Δημιουργία δεδομένου δέντρου διεργασιών} Ο πηγαίος κώδικας της
main.c που γράψαμε είναι ο παρακάτω:
\inputminted[linenos,fontsize=\footnotesize,frame=leftline]{c}{files/ask2-fork.c}

Η έξοδος που προέκυψε κατά την εκτέλεση ήταν η εξής:
\inputminted[linenos,fontsize=\footnotesize,frame=leftline]{bash}{files/ask2-fork.out}
\pagebreak

\emph{Ερωτήσεις}
\begin{enumerate}
\item Αν η διεργασία A τερματίσει πρόωρα τα παιδιά της (B,C) θα συνεχίσουν να
εκτελούνται και υιοθετούνται από την INIT. Όταν αυτά τερματίσουν, η INIT θα
κάνει wait και θα φύγουν από τη μνήμη.
\item Φαίνεται μία ακόμη διεργασία. Αυτή είναι η διεργασία που ξεκίνησε τα
forks δημιουργώντας την A.
\item Σε διαφορετική περίπτωση ένας χρήστης θα μπορούσε να εκτελέσει
απεριόριστες διεργασίες. Κάτι τέτοιο δεν επιτρέπεται, μιας και η ανεξέλεγκτη
κατανάλωση πόρων συστήματος μπορεί να καταστήσει το σύστημά μας μη
λειτουργικό, κάτι που ο διαχειριστής πρέπει να αποφύγει.\\
\end{enumerate}


\subsection{Δημιουργία αυθαίρετου δέντρου διεργασιών} Ο πηγαίος κώδικας
παρατίθεται παρακάτω:
\inputminted[linenos,fontsize=\footnotesize,frame=leftline]{c}{files/ask2-tree.c}

Η έξοδος που προέκυψε κατά την εκτέλεση ήταν η εξής:
\inputminted[linenos,fontsize=\footnotesize,frame=leftline]{bash}{files/ask2-tree.out}
\emph{Ερωτήσεις}
\begin{enumerate}
\item Οι διεργασίες γεννιούνται κατά επίπεδο. Παρόλα αυτά, δεν είναι
εξασφαλισμένη η σειρά που θα εκτελεστούν οι διεργασίες. Έχει να κάνει με τη
χρονοδρομολόγηση (δηλαδή το D του προηγούμενου ερωτήματος θα μπορούσε να
εκτελεστεί πριν από το C αν η χρονοδρομολόγηση το ευνοούσε.
Ωστόσο το κβάντο χρόνου είναι τέτοιο που η εμφάνιση των
μηνυμάτων δημιουργίας μοιάζει σχεδόν σειριακή.
\end{enumerate}

\subsection{Αποστολή και χειρισμός σημάτων} Ο πηγαίος κώδικας
παρατίθεται παρακάτω:
\inputminted[linenos,fontsize=\footnotesize,frame=leftline]{c}{files/ask2-signals.c}
\pagebreak

Η έξοδος που προέκυψε κατά την εκτέλεση ήταν η εξής:
\inputminted[linenos,fontsize=\footnotesize,frame=leftline]{bash}{files/ask2-signals.out}
\emph{Ερωτήσεις}
\begin{enumerate}
\item Είναι σαφώς καλύτερη και πιο αξιόπιστη, καθώς δε στηρίζεται στη
χρονοδρομολόγηση των διεργασιών.
\item Προτού στείλουμε σήματα στα παιδιά πρέπει να έχουμε βεβαιωθεί πως αυτά
είναι σε ready state ώστε να δεχτούν σήμα. Αν το κάνουμε νωρίτερα αυτό θα
αγνοηθεί. Συνεπώς περιμένουμε μέχρι όλα τα παιδιά να κάνουν raise(SIGSTOP)
ώστε να δεχτούν το SIGCONT.
\end{enumerate}

\subsection{Παράλληλος υπολογισμός αριθμητικής έφρασης} Ο πηγαίος κώδικας
παρατίθεται παρακάτω:
\inputminted[linenos,fontsize=\footnotesize,frame=leftline]{c}{files/ask2-pipes.c}

Η έξοδος που προέκυψε κατά την εκτέλεση ήταν η εξής:
\inputminted[linenos,fontsize=\footnotesize,frame=leftline]{bash}{files/ask2-pipes.out}
\emph{Ερωτήσεις}
\begin{enumerate}
\item Κάθε διεργασία χρειάζεται, ένα pipe σε περίπτωση που είναι φύλλο (εκείνο
δηλαδή που θα χρειαστεί για να γράψει στον πατέρα του), ενώ χρειάζεται τρία
στη γενική περίπτωση που είναι κόμβος του δέντρου
(ένα για τον πατέρα του και ένα για κάθε παιδί). Μπορούμε ωστόσο να
χρησιμοποιήσουμε ένα pipe λιγότερο, χρησιμοποιώντας το ίδιο pipe και για τα
δύο παιδιά. Αυτό συμβαίνει διότι στην πράξη του πολλαπλασιασμού και της πρόσθεσης
ισχύει η αντιμεταθετική ιδιότητα. Χρησιμοποιώντας ένα μόνο pipe και για τα
δύο παιδιά, δεν έχω τη δυνατότητα να διακρίνω ποιο παιδί έγραψε τι μέσα στο
pipe. Συνεπώς αν θέλαμε να υποστηρίζουμε και αφαίρεση για παράδειγμα στο δέντρο μας τότε θα
έπρεπε να χρησιμοποιήσουμε ξεχωριστά pipes για να μπορούμε να ξεχωρίσουμε τις
μεταβλητές μας. Επίσης σημαντικός περιορισμός είναι το πλήθος των bytes που
γράφουμε μέσα στο pipe. Δε θα μπορούσαμε να χρησιμοποιήσουμε ένα pipe για να
μεταφέρουμε περισσότερα από 512 bytes, καθώς και πάλι δε θα μπορούσαμε με
βεβαιότητα να διακρίνουμε ποιος έγραψε κάθε 512-δα μέσα στο pipe.
\item Σε ένα σύστημα πολλών επεξεργαστών, σε περίπτωση που ζητήσουμε από μία
μόνο διεργασία να υπολογίσει το δέντρο των μαθηματικών εκφράσεων δεν
εκμεταλλευόμαστε το hardware. Η παράλληλη επεξεργασία θα μας δώσει ταχύτερη
εκτέλεση καθώς κάθε πράξη μπορεί να γίνει ανεξάρτητα απο όλες τις υπόλοιπες.
\end{enumerate}
\end{document}
